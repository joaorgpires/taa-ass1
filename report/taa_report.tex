%----------------------------------------------------------------------------------------
%	CONFIGURATIONS
%----------------------------------------------------------------------------------------

\documentclass[12pt,a4paper,oneside]{article}

\usepackage[utf8]{inputenc}
\usepackage{graphicx}
\usepackage{natbib}
\usepackage{amsmath}
\usepackage{caption}
\usepackage{subcaption}
\usepackage[a4paper,left=2cm,right=2cm,top=2.5cm,bottom=2.5cm]{geometry}
\setcounter{section}{-1}

%----------------------------------------------------------------------------------------
%	INFORMATION
%----------------------------------------------------------------------------------------

\title{Tópicos Avançados em Algoritmos (TAA) - Practical Assignment 1}

\author{João Rebelo Pires\footnote{João Rebelo Pires - 201200384} and José Miguel Oliveira\footnote{José Miguel Oliveira - 201304192}}

\date{DCC - FCUP, April 2017}

\begin{document}

\maketitle

%----------------------------------------------------------------------------------------
%	SECTION 0
%----------------------------------------------------------------------------------------

\section{How To}\label{sec:for_dummies}

In this section, we simply want to explain how to compile and execute our implementation.

\subsection{Input Description}\label{subsec:input_descrip}

The first line of input contains a single integer, an option. If this option is $0$, we are looking for the horizontal partition. Alternatively, the option may be $1$, indicating that we are looking for the grid partition.

The second line of input is an integer, $n\_vertices$, describing the number of vertices the orthogonal polygon has. Then $n\_vertices$ lines follow, the coordinates of the vertices of the polygon, given in counterclockwise order. The coordinates are integer.

The next line of input is an integer, $n\_holes$, describing the number of holes the polygon has. The following lines describe each hole, ina  similar way as the polygon is described.

\subsection{Output Description}\label{subsec:output_descrip}

The output is well represented. It consists on three groups of information.

\begin{enumerate}
	\item The description of each vertex of the DCEL;
	\item The description of each face of the DCEL;
	\item The description of each half edge of the DCEL.
\end{enumerate}

The information each element of the DCEL provides is discussed in section . %TODO \ref para a sec com isto

\subsection{Compilation}\label{subsec:compile}

The following command line compiles the code:

\textit{In MacOS X:}

\texttt{clang++ -Wall main.cpp code/dcelutil.cpp}\\

\textit{In Linux:}

\texttt{c++ -Wall main.cpp code/dcelutil.cpp}

\subsection{Execution}

The compiler generates an executable named \textbf{a.out}. To execute it, simply execute the following command line:

\texttt{./a.out}.\\

If you have an input file \textbf{test.in}, you should execute the following command line instead:

\texttt{./a.out < test.in}.

%----------------------------------------------------------------------------------------
%	SECTION 1
%----------------------------------------------------------------------------------------






%----------------------------------------------------------------------------------------
%	SECTION 2
%----------------------------------------------------------------------------------------





%----------------------------------------------------------------------------------------
%	SECTION 3
%----------------------------------------------------------------------------------------






\end{document}
